\section{Муравьиные алгоритмы.}

Муравьиные алгоритмы (МА) основаны на использовании популяции потенциальных решений (это EDA на графах) и разработаны для решения задач комбинаторной оптимизации, прежде всего, поиска различных путей на графах. 

Кооперация между особями (искусственными муравьями) здесь реализуется на основе моделирования стигмергии. При этом каждый агент, называемый искусственным муравьем, ищет решение поставленной задачи. Искусственные муравьи последовательно строят решение задачи, передвигаясь по графу, откладывают феромон и при выборе дальнейшего участка пути учитывают концентрацию этого фермента. Чем больше концентрация феромона в последующем участке, тем больше вероятность его выбора.

Реальные муравьи благодаря стигмергии способны находить кратчайший путь от гнезда до источника пищи достаточно быстро и без визуального (прямого контакта). Более того, они способны адаптироваться к изменениям окружающей среды.

Стигмергия – непрямая форма общения муравьев, основанная на том, что муравьи по пути движения откладывают специальный фермент – феромон.

Примеры:
\begin{itemize}
      \item Ant System
      \item Ant Colony System
      \item Min-max Ant System
   \end{itemize}

