\section{Декартово ГП. Семантическое ГП.}

\subsection*{Декартово ГП.}
Идея: структуры программы дана, нужно лишь заполнить ее операторами и ссылками на переменные.

Общий вид:
\begin{figure}[h]
\centering
\includegraphics[width=0.8\linewidth]{images/decart.png}
\caption{Пример}
\label{fig:mpr}
\end{figure}

Ограничение в изначальном наборе операторов.

\subsection*{Семантическое ГП.}

Основная идея состоит в том что мы хотим изменять не гены которые могут изменять сильно фенотип а непосредственно сам фенотип. 
Для этого используются следующие подходы:
\begin{itemize}
	\item Обертки (Wrappers). Стандартные операторы, проверяем семантику, при необходимости повторяем.
	
	При необходимости: либо тождественно родителю, либо сильно отличается
	\item Операторы, которые в курсе семантики.
	
	Пример: обмен только семантически схожих (но не идентичных) поддеревьев
	\item Инициализация, которая в курсе (Semantic-aware initialization). Порождает семантически разнообразные начальные популяции.
	
	Пример: если поведение особи похоже на уже имеющееся, пересоздаем.
	\item «Геометрические» операторы. 
	\begin{itemize}
		\item Мутация: семантика внутри окрестности семантики родителя
		\item Скрещивание: семантика на отрезке между семантикой родителей
		\item Определения окрестности и отрезка зависит от используемой метрики
		\item Преимущество: Оптимизация делается просто, так как функция приспособленности упрощается или даже становится унимодальной
		\item Недостаток: высокий рост размера программы
	\end{itemize}
\end{itemize}
