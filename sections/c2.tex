\section{Постановки задач оптимизации. Общие черты «эволюционных» постановок.}
\begin{center}
     \textbf{Комбинируемые разновидности постановок\\}
\end{center}
\begin{itemize}
\item Минимизация или максимизация одной функции\\
- Пространство поиска: дискретное, непрерывное, смешанное\\
- Функция: «черный ящик», «серый ящик»
\item Оптимизация в ограничениях\\
- Вид: строгие (равенство), нестрогие (неравенство)\\
- Способ задания: аналитически, «черный ящик», …
\item Мультимодальная оптимизация\\
- Поиск нескольких/всех оптимумов функции
\item Оптимизация зашумленных функций \\
- Устойчивость к шуму в измерении приспособленности
\item Оптимизация меняющихся со временем функций\\
- Изменения могут сообщаться или не сообщаться алгоритму
\item Многокритериальная оптимизация\\
- Несколько функций, каждую из которых требуется минимизировать/максимизировать\\
- Функции конфликтуют → требуется найти много репрезентативных решений
\item Многоуровневая оптимизация\\
- Задача оптимизации, в ограничениях которой — другая задача оптимизации
\item Одновременная оптимизация (мультитаскинг)\\
- Есть несколько похожих задач, хотим решить их все за один запуск\\
- Может быть быстрее, чем решать по отдельности
\item Коэволюция\\
- Приспособленность может быть вычислена только при состязании или кооперации с другими особями \\
- ... и многое другое!\\
\end{itemize}
\begin{center}
     \textbf{Общие черты «эволюционных» постановок.\\}
\end{center}
 \textbf{Решение задач оптимизации в следующих условиях:}
 \begin{itemize}
\item Задачи — сложные (как правило, NP-трудные) нет смысла решать простые задачи
\item Постановки задач — в меру формальные\\
- Мера качества решения задачи известна — но может быть очень сложной (например, результаты отработки в симуляторе)
\item Стабильность постановки задач — низкая\\
- В любой момент могут быть даны новые вводные\\
- Постановка задачи может измениться после пробных запусков
\item Времени на разработку и улучшение алгоритма — мало
\item Получающиеся решения – приближенные, без гарантий качества\\
\end{itemize}

\textbf{Метафора №1: Эволюционные алгоритмы — методы решения задач плачущими инженерами} \\
При таком подходе нет константного результата, нет гарантированного результата, не подразумевается эффективное решение. \\
\textbf{Метафора №2: Evolutionary algorithms are the second best approach to solve any problem\\}
