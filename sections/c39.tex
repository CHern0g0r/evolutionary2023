\section{Методология сравнения результатов работы.}

Что можем сравнивать:
\begin{itemize}
    \item Число вычислений функции приспособленности для
    нахождения оптимума.
    \item Число вычислений функции приспособленности для
    достижения заданного уровня приспособленности.
    \item Приспособленность лучшего решения при заданном
    вычислений фитнесса.
    \item Средняя приспособленность популяции после заданного
    числа вычислений фитнесса.
    \item ...
\end{itemize}

Результат работы - выборка значений случайной величины.

\begin{itemize}
    \item Недостаточно запустить алгоритм один раз.
    \item Усреднить тоже не всегда достаточно.
    \item \textbf{Используем статистические тесты!!!}
    Часто берем непараметрические тесты, т.к. распределения
    не нормальны (Уилкоксон).
\end{itemize}

Сравнивать алгоритмо стоит на множестве задач.

Проблема: если $A > B$ на 1 тесте (на остальных равны), то это
не значит, что $A > B$ на самом деле, т.к. рандом.

Решать эту проблему можно процедурами коррекции, но что это такое,
никто рассказать не удосужился.
