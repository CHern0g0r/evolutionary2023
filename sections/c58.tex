\section{IBEA. Many-Objective Optimization.}
Indicator-Based Evolutionary Algorithm (IBEA)\\

Свойства:\\
- Использует бинарный индикатор I(·,·)\\
- Устанавливает скалярную «приспособленность» решениям в популяции следующим образом: \\
$F(x) = \sum_{y \in P/(x)} \mp e^{I(y)(x)/k}$.($I(y)(x)/k$ - отношение бинарного индификатора к вектору критериев, $y$ - разница между доминирующем множеством и заданным $x$)\\
- Если семантика $F(x)$: «потеря качества», значит что $x$ будет удалено из популяции\\

Алгоритм:\\
- Поддерживаем множество недоминируемых решений размера не более $N$\\
- Создаем $N$ новых решений, добавляем их все в популяцию\\
- Пока в популяции больше N решений, удаляем решение с минимальным F.\\

Здесь работает механизм семантического отбора, который исчисляет значение индикаторной функции для недоминирующего множества. Таким образом нам известно каждое значение x с y, что позволяет нам убирать те особи, которые имеют максимальную потерю качества\\

Many-Objective Optimization\\
Определения понятию нет, просто «слишком много критериев»\\
- Некоторые считают, что $K \geq 4$ — это «слишком много» (Некоторыми критериями можно пренебречь, не все они являются значимыми)\\
Общее свойство таких постановок задач: два случайных решения, скорее
всего, не будут доминировать друг друга и будут друг от друга отличаться радикально\\
- Простой случай: выборка из гиперкуба ($x_{i} ← UniformRandom[0; 1]$). Берём подмножества из гиберкуба и получаем, что все они не доминируют друг друга и имеют различия по многокритериальному вектору\\
- Вероятность какого-то доминирования: $2 · (\frac{1}{2})^K$ : сходится к нулю с ростом K\\
- Аналогично с расстоянием между точками: неограниченно растет с ростом $K$. Два различных решения, которые не коррелируют и не доминируют из-за большого количества критериев\\
- Чем больше K, тем меньшее давление отбора возможно при использовании «естественных» величин, определяющих разнообразие. Большое количество критериев снижает эффективность предыдущих механизмов отбора, т.к. мы не можем удовлетворить сразу все критерии  \\
- Приходится использовать величины неестественные и сверхъестественные (Говорится об индикатарах, рангах и др.) 
