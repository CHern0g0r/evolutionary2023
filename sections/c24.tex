\section{Генетическое программирование.}
Идея состоит в выращивании программ (не код на питоне а что то более абстрактное) путем эволюции

Для описания можно использовать:
\begin{itemize}
\item Символьная регрессия
\item Конечные автоматы-распознаватели
\end{itemize}
Также мы можем представлять программы в виде деревьев:

Особи=решения: деревья разбора программ.

Приспособленность: основана на запуске закодированной программы. 
 \begin{figure}[h]
\centering
\includegraphics[width=0.8\linewidth]{images/evolv_gen.png}
\caption{Пример}
\label{fig:mpr}
\end{figure}

Дерево состоит из следующих компонентов: Функции нулевой арности (листья) и Функции ненулевой арности (внутренние вершины).

Функции нулевой арности могут быть:
\begin{itemize}
	\item Константы
	\item Переменные
	\item Эфемерные константы: генерируются в момент первого обращения
	\item Параметры: можно дополнительно настраивать при вычислении приспособленности
	\item Вызовы других деревьев (функции)
\end{itemize}

Функции ненулевой арности (внутренние вершины) могут быть:
\begin{itemize}
	\item Арифметические и логические операции
	\item Управляющие операторы (if-then-else, while и тд )
	\item Произвольные функции
\end{itemize}
Узлы дерева также могут иметь тип что также надо учитывать. 

Инициализация дерева:
\begin{enumerate} 
	\item Необходимо ставить ограничения на глубину решения 
	\item Grow: глубина каждого листа не превышает максимальную
	\item Full: глубина каждого листа равна максимальной
	\item Ram pedHal fAndHal f: с вероятностью p = 0.5 вызываем в поддереве Grow, иначе Full
	\item Можем балансировать между максимальной глубиной листа
\end{enumerate}

Мутации:
\begin{itemize}
	\item Заменить поддерево случайно сгенерированным деревом
	\item Заменить внутреннюю вершину одним из ее детей
	\item Заменить операцию во внутренней вершине случайной операцией той же арности
	\item Поменять детей внутренней вершины, если в этом есть смысл
	\item Поменять два случайно выбранных поддерева
	\item Мутировать константу в листе
	\item чето еще
\end{itemize}

Кроссовер:
\begin{itemize}
	\item Как правило, обмен поддеревьями
	\item Часто встречающаяся рекомендация: с p = 0.1 выбирать для обмена лист, иначе внутреннюю вершину
	\item Может радикально изменить фенотип, но при обмене похожими поддеревьями «можно жить»
\end{itemize}

Можем генерировать программы используя стековые языки (примитивные инструкции).
Соотвественно тогда для решения нам не нужны деревья а просто список инструкций переменной длины.

В случае с достаточно мощными языками программы могут генерировать и запускать код, заниматься самомодификацией
