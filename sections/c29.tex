\section{Сводка по методам теоретического анализа ЭА.}

\begin{itemize}
    \item \textbf{Точный подсчет распределений с использованием цепей Маркова} =
    описать распределение состояний ЭА после t шагов.
    \item \textbf{Доказательство сходимости к оптимуму} = ЭА сходится к нужному
    оптимуму "любой"\ задачи (Без описания затраты ресурсов).
    \item \textbf{Теория схем} = ГА эффективен, потому что умеет обрабатывать
    схемы. Вроде речь идет о вычислительных схемах из теории сложности и вроде
    как это описывает класс задач, которые можно решать. (Но это я так услышал)
    \item \textbf{Анализ времени работы} = доказываем среднее время (количество
    запросов) работы алгоритма до нахождения оптимума.
    \item \textbf{Анализ при фиксированном вычислительном бюджете} = фиксируем
    t - количество запросов, доказываем среднюю асимптотику приспособленности
    при стольких запросах.
    \item \textbf{Вычислительная сложность задач оптимизации} = показываем
    среднее количество запросов для решения задач оптимизации из фиксированного
    класса нашим ЭА.
\end{itemize}