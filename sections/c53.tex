\section{Ранжирование решений.}
Ранжирование решений в алгоритме SPEA2 (Dominance count)\\
Фаза 1: Вычисление «силы» недоминированных особей\\

- Для любого недоминированного решения. $S$ - функция силы, $\frac{1}{n}$ - обратное соотношение мощности к популяции, $y$ - выборка решений (множество доминируемых решений) $x, S(x) = \frac{1}{n}·|{y | x \leq y}|$\\
- $S(x) \geq \frac{1}{n}$, так как $x \leq y$\\
- Чем меньше $S(x)$, тем больше мы теряем, удалив $x$\\

Фаза 2: Вычисление меры качества всех особей\\
Пусть $P_{0}$ — множество недоминированных решений\\
- Для каждого решения $y$ вычисляется следующая величина:\\
- $F(y) = \sum_{x \in P_{0}|x \leq y} S(x)$\\
- Если $x \prec y$, то $F(x) \leq F(y)$\\
- Меньшие значения F соответствуют менее населенным регионам (область популяции)\\

В общем - сначала определяем значение силы для недоминирующих особей ($x$), далее происходит определение меры качества всех особей $(F(x)$ и $F(y))$, затем на основе полученных значений $S(x)$ происходит ранжировка на доминирующие и недоминирующие\\

Ранжирование решений, равных по основному рангу\\

Нужно уметь различать решения, равные по основному рангу,
основанному на доминировании.\\

Методы:\\
-Информация о разнообразии в популяции в окрестности решения\\
1) Расстояние до ближайшего соседа\\
2) Расстояния до нескольких ближайших соседей\\
3) Сколько решений находится внутри определенных гиперкубов\\
Благодаря данной информации можно составить некоторый ранг, то есть, в первом случае - функция расстояния ближайших соседов. Это дает нам оператор, с помощью которого мы определяем принцип ранжировки\\

-Вклад решения в меру качества популяции. То есть есть некоторая величина, характеризующая качество популяции на данном этапе и она зависит от каждого решения индивидуально (индивида). При удалении индивида из популяции мыможем оценить насколько он был значим в функции коллективной ответственности, то есть оценить его вклад в качество популяции\\
1) Насколько качество популяции уменьшится, если решение будет удалено?\\
2) Индикаторы: следующий раздел - функции, которые выводят о качество решения в популяции (сколько процентов элиты, сколько новых решений)
