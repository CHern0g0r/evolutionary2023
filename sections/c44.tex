\section{Кооперативная коэволюция.}
\textit{смотри определение в 42 вопросе}\\
Пример применения: декомпозиция решения, «разделяй и властвуй»
\begin{itemize}
    \item Разбиение как задачи, так и ее решений, на смысловые части
    \item Разные оптимизаторы занимаются оптимизацией различных типов частей
    \item Приспособленность: собираем решение из частей и обсчитываем его адекватности для текущей задачи
\end{itemize}

\textbf{Стратегии вычисления приспособленности:}
\begin{itemize}
    \item Последовательная: выбираем особь, затем k раз:
    \begin{itemize}
        \item Выбираем случайных коллабораторов из каждой популяции
        \item Собираем решение и вычисляем его приспособленность\\
        Есть некоторая текущая декомпозиция, грубо говоря, есть процесс решения задачи, он затрагивает 10 шагов, т.е. 10 частей. Выбираем особь, которая представляет собой комбинацию из 10 частей, затем k раз выбираем случайных операторов. Т.е. у нас каждый раз новая комбинация этих самых частей. И далее из этой комбинации вычисляется приспособленность для текущей задачи 
    \end{itemize}
    \item Параллельная: выбираем особь, пока она не посчиталась k раз
    \begin{itemize}
        \item Выбираем случайных коллабораторов из каждой популяции
        \item Собираем решение и вычисляем его приспособленность\\
        Вычисления приспособленности происходит параллельно со сбором частей. Т.е. мы взяли 10 операторов путем применения операторов мутации и скрещивания получили еще 10х10 операторов. И каждый раз собираем по 10ке, т.е. у нас происходит не последовательный перебор, а параллельный т.е. вычисление приспособленности происходит для каждого набора случайно выбранного в себе.
    \end{itemize}
\end{itemize}
\textbf{Как переносить приспособленность решения на составляющих ее особей?}\\
У нас есть некоторое решение - тот самый набор из 10 операторов, у нас есть значения функции приспособленности для этого решения (тех 10ти операторов). Как перенести ее на те самые операторы. Каждый оператор должен иметь какую-то функцию значимости, благодаря которой он выбирается в лучшую декомпозицию решения.
\begin{itemize}
    \item Усреднение по решениям: Средние результаты лучше, максимальные хуже ( хорошо когда частей не много). Отделяно взятая особь с максимальным значением функции присвособленности не будет считаться лучшей, поскольку учитывается коллективная ответственность композиций частей.
    \item Максимум по решениям: Понадобятся большие значения k (для большого количества  частей)
\end{itemize}
