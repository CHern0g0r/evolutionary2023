\section{Модельные задачи ЭВ. Бенчмарки.}

\subsection*{Модельные задачи}

Как доказать полезность ЭА = модельные задачи.

\D{
    Модельные задачи = набор "стандартных задач для предметных"
    обрастей. Разработаны специально для сравнения алгоритмов.
    Пытаются покрывать основные сложности, характерные для
    предметной области.
}

Преимущества:
\begin{itemize}
    \item Есть с кем сравниваться
    \item Можно делать выводы о недостатках и на вашей задаче тоже.
    \item Дает путь к исправлению с помощью анализа модельной задачи.
\end{itemize}

Недостатки:
\begin{itemize}
    \item Централизованный контроль качества отсутствует.
    Некоторые используют не все задачи из набора и искажают
    формулировки задач.
    \item Сообщество переобучается под конкретные наборы.
\end{itemize}

\subsection*{Бенчмаркинг}

Правильный подход для исследования алгоритмов:
\begin{itemize}
    \item Запускать много раз!
    \item Анализировать статистику, а не только лучший результат.
    \item Стоит смотреть не только на окончательный результат, но
    и на процесс.
\end{itemize}
