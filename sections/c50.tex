\section{Виды алгоритмов многокритериальной оптимизации.}
1) Основанные на объединении критериев (Aggregation-based) - критерии сводятся к одному, который в дальнейшем и оптимизируется. \\
Если не получается свести к одному критерию, то сводим к множеству равнозначных критериев, которые сопоставимы между собой.

2) Основанные на отношении доминирования (Domination-based).\\
Ряд критериев располагаются в доминирующем порядке, как и их решения (согласно доминированию критериев), что позволяет избежать одинаковости решений.

3) Основанные на использовании индикаторов (Indicator-based). \\
Сведение к однокритериальной оптимизации на наборах решений путем оценки или сравнения качества наборов решений. Необходима стороняя метрика для сравнения качества решений по ряду критериев. Вместо оценки оптимума по множеству критериев появляется индикаторная функция которая задает оценку качества решений. Сама функция включает в себя зависимость от этих критериев, то есть она говорит удовлетворяет ли решение критериям или нет 

4) Основанные на декомпозиции (Decomposition-based).\\
Сведение к множеству различных однокритериальных задач оптимизации путем объединения критериев различными способами.

Самый эффективный из представленных четырех - первый алгоритм. Правильная агрегация позволит свести оптимизацию к однокритериальной, которая сохранит смысловую нагрузку критериев, решения будут независимыми от способов выбранных способов агрегации
