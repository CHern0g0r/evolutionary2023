\section{Оптимизации с ограничениями.}

\subsection{Определение}
\begin{itemize}
    \item Дано $x \in X \text{, где X: (просто определённое) пространство поиска} $
    \item Минимизировать $f_{1}(x) \dots f_{n}(x) \text{ возможность многокритеальной оптимизации} $
    \item При условии:
    \begin{itemize}
        \item $g_{1}(x) \leq 0 \dots g_{m}(x) \leq 0 \text{ -  нестрогие ограничения} $
        \item $h_{1}(x) = 0 \dots h_{k}(x) = 0 \text{ - строгие ограничения} $
    \end{itemize}
\end{itemize}

\subsection{Основные методы}

\begin{enumerate}
    \item \textbf{Штрафные функции}
    \subitem Ухудшать приспособленность особей, не удовлетворяющих ограничениям
    \item \textbf{Специальные представления и операторы}
    \subitem  Использовать представления и операторы, которые по построению создают особи, удовлетворяющие ограничениям
    \item \textbf{Починка особей}
    \subitem  Если особь не удовлетворяет ограничениям, изменить ее
    \item \textbf{Разделение ограничений и целевых функций}
    \subitem Даже не пытаться учитывать ограничения путем модификации функции приспособленности
\end{enumerate}
