\section{Штрафные функции.}

\subsection{Штрафные функции: Основные разновидности}
\begin{enumerate}
    \item \textbf{Exterior penalty}
    \begin{itemize}
        \item Удовлетворяем ограничениям: приспособленность не меняется
        \item Не удовлетворяем: приспособленность ухудшается пропорционально степени выхода за границы
    \end{itemize}


    \item \textbf{Interior penalty}
    \begin{itemize}
        \item  Удовлетворяем ограничениям: приспособленность ухудшается пропорционально близости к границе
        \item  Не удовлетворяем: считаем одинаково плохими
    \end{itemize}

    \item Типичный сценарий для метаэвристик $\varphi(x) = f(x) + ^{\sum m} _{i=1} r_{i} \cdot G_{i}(g_{i}(x)) + ^{\sum k} _{j=1} s_{i} \cdot H_{i}(h_{i}(x))$
    \item Типичные функции \begin{gather*}
                               G_{i}(v) = \max(0, v)^{\alpha}\\
                               H_{i}(v) = |v|^{\beta}\\
                               \alpha, \beta = 1 \text{ или } 2\\
    \end{gather*}
    \item Ослабление строгих ограничений $|h|_{j}(x) - \sigma \leq 0$
\end{enumerate}


\subsection{Простые штрафные функции}

\begin{enumerate}
    \item \textbf{"Смертельный" штраф}
    \begin{itemize}
        \item [~] Любое решение, не удовлетворяющее ограничениям, бесконечно плохое
        \item [+] очень легко реализовать
        \item [--] отсутствие градиента за пределами границ
        \item [--] плохо работает, если множество допустимых решений мало или несвязно
    \end{itemize}

    \item \textbf{Статический штраф}
    \begin{itemize}
        \item [~] Штрафные коэффициенты $r_i, s_i$ зафиксированы на протяжении всей оптимизации
        \item [+] очень легко реализовать
        \item [--] может сильно зависеть от задачи
        \item [--] нестатический выбор может быть (доказуемо) лучше
    \end{itemize}
\end{enumerate}

\subsection{Нестатические штрафные функции}
\begin{enumerate}
    \item \textbf{Динамический штраф}
    \begin{itemize}
        \item [~] Штрафные коэффициенты (полиномиально) растут со временем оптимизации. Например: $(C * t)^\alpha$ для t-ой
        итерации, $\alpha \in [1; 2]$
        \item [+] весьма легко реализовать
        \item [+] могут работать лучше «из коробки»
        \item [--] все еще зависит от задачи
        \item [--] может быть трудно найти правильную формулу (как в методе отжига)
    \end{itemize}

    \item \textbf{Адаптивный штраф}
    \begin{itemize}
        \item [~] Общий штрафной коэффициент $\lambda$ адаптируется на основании истории поиска:
        \item [~] Гиперпараметры: $\beta1 > \beta2 > 1$ (общие), k (зависящий от задачи)
        \item [+] управляет штрафами в зависимости от того, что сейчас происходит
        \item [+] хорошо работает «из коробки»
        \item [+] нужно настраивать меньше параметров для каждой новой задачи
        \item [--] хочет, чтобы популяция содержала и «хорошие», и «плохие» особи
        \item [--] все еще нужно настраивать k
    \end{itemize}
\end{enumerate}
