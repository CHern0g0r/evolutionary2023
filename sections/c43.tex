\section{Соревновательная коэволюция с 1 и 2 популяциями.}
\textit{смотри определение в 42 вопросе}
\subsection{Соревновательная коэволюция с одной популяцией}
Пример применения: эволюция игровых стратегий\\
\textbf{Мотивация}
\begin{itemize}
    \item Большинство случайных стратегий слишком плохи, чтобы отбор по абсолютной приспособленности заработал.
    \item Пусть они сражаются друг с другом больше информации из исхода игры
    \item Со временем они вырастут над собой (и убьют всех людей)\\
    Как правило имея 10 случайных стратегий в играх, только 1-2 из них являются лучше остальных. Т.е. порядок хороших стратегий в общем случае крайне мал. В таком случае логичнее всего построить модель какого-либо генетического алгоритма, который будет иметь в себе соревновательную коэволюцию. Т.е. если у нас не существует понимания какая из стратегий является лучшей, то проще внедрить соревновательный механизм между большинством игровых стратегий. На каждой итерации получим значение, которое будет определять эффективность той или иной стратегии. Изначально у нас нет этого значения, но с течением итераций будем ориентироваться на него. Что-то вроде искусственного введения показателя качественности стратегий.
\end{itemize}

\textbf{Как можно устраивать сражения?}\\
    Можно делать какой-либо турнирный коэффициент, либо кадый с каждым и прогонять ту самую стратегию в контексте игры.\\
    Варианты комбинаций соревнований:\\
\begin{itemize}
    \item Каждый с каждым: $\frac{N(N-1)}{2}$ игр на итерацию
    \item Разбить на пары случайным образом: $\frac{N}{2}$ игр
    \item Выбрать $K$ случайных соперников: $NK$ игр
    \item Лучше всего - турнирная, т.к. предыдущий вырождаются в чистый перебор. Если количесто итераций будет слишком велико, то мы просто исчерпаем механизм случайности.
    \item Один большой турнир: $N-1$ игр, хорошие решения проведут больше игр и точнее посчитаются
    \item Отбор без явной приспособленности: просто запускаем игры, их результат и будет использоваться компаратором.(на большом количестве стратегий и итераций также будет вырождаться в полный перебор) отлично подходит, когда мало стратегий и мало итераций
\end{itemize}
\subsection{Соревновательная коэволюция с двумя популяциями}
Пример применения: генерация программ против набора тестов\\
\textbf{Оптимизаторы для разных популяций могут быть разными!}
\begin{itemize}
    \item Первая популяция: особи, являющиеся решениями исходной задачи оптимизации
    \item Вторая популяция: особи-тесты, которые со временем станут сложными экземплярами задачи
    \item т.е. с одной стороны у нас есть решение исходной задачи, с другой стороны - оценка эффективности решений.
    \item Классический пример: построение генетическим алгоритмом сортирующих сетей ( Hillis, 1990)
\end{itemize}
\textbf{Вычисление приспособленности:}
\begin{itemize}
    \item Состязание с текущим поколением соперников
    \item ...или с предыдущим поколением соперников
\end{itemize}

\textbf{Возможные проблемы:}\\
Тесты усложняются слишком быстро потеря градиента (все решения «плохие») и тормозит итератиынй процесс. Может быть так, что весь генетический алгоритм будет работать быстрее, чем оценка эффективности решений(тесты) и при таком раскладе, все решения - плохие, потому что в одном случае функция применимости не вычисляется за заданное время, в другом - неприемлемые значения.
