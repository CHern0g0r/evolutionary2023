\section{OneMax. RLS на OneMax. 1+1 на OneMax. Оценки вычислительной сложности OneMax.}

\D{
    \textbf{OneMax} - максимизация числа единиц в битовой строке.

    $OneMax(x) = \sum\limits_{i=1}^n x_i \to MAX;\; x \in \{0, 1\}^n$

    $\Leftrightarrow$

    $OneMax_n: \{0, 1\}^n \to \mathbb{R}; x \mapsto |\{i \in [1..n] | x_i = 1\}|$

    Усложненная версия: угадывание загаданной строки

    $OneMax_n: \{0, 1\}^n \to \mathbb{R}; x \mapsto |\{i \in [1..n] | x_i = s_i\}|$
}

\subsection*{RLS}

\D{
    RLS алгоритм:
    \begin{itemize}
        \item Поддерживается популяция из одной особи
        \item Мутация: инвертируется случайный бит
        \item Если мутант не хуже родителя - заменяем особь
    \end{itemize}
}

\T[OneMax with RLS]{
    $f$ - приспособленность, $n$ - длина строки.

    $P[\text{инвертируется правильный бит}] = \frac{n - f}{n}$

    $\mathbb{E}[\text{число запросов до правильной инверсии}] =
    \frac{n}{n - f}$

    $\mathbb{E}[T(n)] = \sum\limits_{f=0}^{n-1} \frac{n}{n-f} =
    n \cdot \sum\limits_{i=1}^n \frac{1}{i} = n\ln n + \Theta(n)$
}

\subsection*{1+1}

\D{
    (1+1) алгоритм:
    \begin{itemize}
        \item Поддерживается популяция из одной особи
        \item Мутация: каждый бит инвертируется с вероятностью
        $\frac{1}{n}$
        \item Если мутант не хуже родителя - заменяем особь
    \end{itemize}
}

\T[OneMax with (1+1)]{
    $P[\text{инверсия ровно 1 бита}] = P_1 = (1 - \frac{1}{n})^{n-1}
    \geq \frac{1}{e}$

    $P[\text{инвертировали правильный бит} | P_1] = \frac{n-f}{n}$

    $\mathbb{E}[\text{число итераций до улучшения}] \leq \frac{en}{n-f}$

    $\mathbb{E}[T(n)] \leq e n \ln n + O(n)$

    $\mathbb{E}[T(n)] \leq e n \ln n - \Theta(n)$
}

\subsection*{Оценки сложности OneMax}

Нижняя оценка: $\frac{n}{\log_2 n} + 1 - o(1)$

Эта штука выводится из энтропии и теории информации.

Следующая формула более точная:
$\log_n(1 + 2^n (n-1)) - \frac{1}{n-1}$

Верхняя оценка: $\frac{(2 + o(1)) \cdot n}{\log_2 n}$

Дальше какая-то непонятная ересь, еще и знаки из формулы пропали:

t независимых случайных запросов согласуются с единственной
загаданной битовой строкой с вероятностью $1 - o(1)$, если:

$t \geq 1 + \frac{4 \log \log n}{\log n} \cdot \frac{2n}{\log n}$