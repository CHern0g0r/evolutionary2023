\section{Модельные задачи оптимизации.}

\D{
    \textbf{OneMax} - максимизация числа единиц в битовой строке.

    $OneMax(x) = \sum\limits_{i=1}^n x_i \to MAX;\; x \in \{0, 1\}^n$

    $\Leftrightarrow$

    $OneMax_n: \{0, 1\}^n \to \mathbb{R}; x \mapsto |\{i \in [1..n] | x_i = 1\}|$

    Усложненная версия: угадывание загаданной строки

    $OneMax_n: \{0, 1\}^n \to \mathbb{R}; x \mapsto |\{i \in [1..n] | x_i = s_i\}|$
}

\textbf{LeadingOnes} - максимизация длины префикса из единиц в
битовой строке.

OneMax можно обобщить в игры на угадывание.
\begin{itemize}
    \item Есть неизвестная битовая строка s длины $n$.
    \item Способ отгадывания = запросы к оракулу, который возвращает $\sum [x_i = s_i]$
    \item Задача = угадать строку за минимальное число запросов.
\end{itemize}

Взвешивание монеток:
\begin{itemize}
    \item n настоящих, m фальшивых монет разного веса.
    \item Найти все настоящие монеты за минимальное число взвешиваний.
\end{itemize}
